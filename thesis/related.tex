When dealing with mobile phones, the first inclination may be to avoid servers entirely and try to construct an ad-hoc network. Research exists that uses virtual certificate authorities that allow mobile nodes to issue certificates (and thereby, confer trust) to other nodes. In such networks, link breakage becomes a concern, threatening reliability. In addition, these ad-hoc networks become quite vulnerable to attack after extended use, suggesting that at least some kind of centralized server must be used\cite{5972278}.

The problem of verifying a user's identity and trust level is a common one. In several cases this is accomplished through a capsule or token that is conferred to the requester once they are confirmed as trusted\cite{6040529}\cite{fongen1}\cite{fongen2011federated}. Encrypted usernames and passwords allow a user to access records independently of a device; however, usernames and passwords are possible to leak and must be guarded very carefully\cite{6115545}. X.509 ties identity to certificates, so information is not as accessible as with a username and a password, but is reasonably secure as long as the certificate authority is trusted\cite{4205196}\cite{6115545}\cite{5972278}\cite{fongen2011federated}.

Trust is a much more complicated subject than merely having or not having trust. Different levels of trust can also be employed to make sure that a user only has access to needed resources. Protocols exist to negotiate these different levels, as well as automatically rate the trust level of a requester. However, it is also important to remember than in a true emergency, the patient's life should always take priority over trust levels\cite{6198123}.

From these works, it is fairly clear that the main difficulty for any protocol is making sure that the trust negotiation is robust and secure. Once trust is secured by an individual, the delivery of records needs to be protected. In addition, any protocol that tries to tackle this problem on mobile phones needs to be fast and efficient enough that it can be ran on mobile smartphones.