As shown in Figures 3 and 4, smartphone processors have evolved enough that the encryption steps in this protocol are fairly trivial in terms of time taken. Even the X.509 certificate validation used in SSL completes fairly quickly, and the bandwidth consumed over the network calls is trivial for such small Java objects. The largest concern is in regards to scalability --- not many Trust Server hops were tested while executing the prototype due to infrastructure constrains. A countrywide deployment of the protocol would likely require many more Trust Server hops and take much longer to locate the correct Trust Server responsible for signing the identity token. This could be mitigated by increasing the number of neighbors each Trust Server begins with, although this may become prohibitively expensive to set up.

There are still weaknesses in the protocol that were made apparent during development of the prototype that need to be addressed before it is truly viable for real world usage (these weaknesses will be discussed further in the next section). Even once these weaknesses are addressed, a protocol responsible for such sensitive information as health records should have the utmost security. The proposed protocol would have to be examined much more closely to even begin consideration for use in the wild.

Even assuming the protocol has impenetrable security (an impossible feat), the myriad of regulations it would need to meet to become official likely prevent its use. Still, the successes (and failures) during development serve as a valuable lesson to be used when making protocols of this nature, and serve as a good starting point for future works.