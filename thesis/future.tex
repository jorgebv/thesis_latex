There are a few places the current protocol needs refinement before extensions are considered. First, the method in which Trust Servers locate the patient that is being requested in not clearly defined. In the prototype, this issue is avoided and left up to future implementations to handle. It is assumed that each Trust Server is able to easily look up whether a patient is located in the Health Record Server(s) that the Trust Server is responsible for. Beyond this, something similar to the routing algorithms used in network routers may prove effective. Initial searches during setup may prove time consuming, but if each Trust Server is able to maintain a table of patients and paths through which this patient can be reached, the time cost should stabilize after enough queries. This also has the added benefit of allowing Trust Servers and Health Record Servers to be added to the system without having to notify and reconfigure the entire infrastructure.

The other point of concern is the symmetric key that is used to encrypt communication between the requester and the Health Record Server. In the prototype, the key generation step is skipped and the same key is used for all communication. Although this measures the costs of encryption and decryption, it is not a secure solution for the long term. There are several options here, and key generation and delivery is a well studied problem. A variant of Diffie-Hellman could be used without adding much complexity to the protocol, as long as one is careful to protect against attacks. Or the already encrypted SSL connections of the Trust Servers could be used to deliver unique keys for each session.

There are also many interesting ideas that could be further research for future work on the protocol (or other protocols entirely). The trust mechanisms could be made more robust so as to handle different roles and access restrictions rather than trusting requesters with full access to health records immediately, as discussed in \cite{6040529}.

There are also a myriad of different cryptographic schemes and delivery methods that could be leveraged throughout the protocol. Rather than relying on the arbitrary restrictions imposed by program implementations, perhaps SIP could be leveraged to negotiate the protocols to be used. However, this would add another point of failure in the protocol by the necessity of introducing SIP servers \cite{sparks2007sip}.